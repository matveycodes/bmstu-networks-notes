\section{Лекция №5 (16.10.2023)}

\subsection{CIDR}

В CIDR разрешено перемещать маску влево и вправо (в отличие от VLSM, где только вправо).

<Рисует задачу на CIDR>

\subsection{VLAN}

Оборудование на сетевом уровне может прочитать заголовок и из него извлечь IP-адреса. Оборудование на канальном уровне (в частности коммутатор) этого делать не умеет.

Чтобы оперировать понятиями сетей и подсетей на канальном уровне придумали VLAN (Virtual Local Area Network~--- виртуальная локальная сеть). Стандарт IEEE 802.1Q. В стандартный кадр Ethernet включается поле Tag (4 байта)~--- номер VLAN.

\image
{\textwidth}
{05/notes/inc/vlan-frame}
{Тэгированный кадр Ethernet II}

По умолчанию всегда существует первый (дефолтный, нативный) VLAN. Остальные надо создавать.

Конечные устройства не понимают и не отправляют тэгированные кадры. В тот момент, когда кадр приходит на интерфейс коммутатора, тот добавляет поле Tag в кадр. Кадры в коммутаторе всегда (кроме кадров нативного VLAN) тэгированы. При выходе с интерфейса коммутатора тэг снимается.

\image
{.75\textwidth}
{05/notes/inc/vlan-topology}
{}

Если интерфейс, на который нужно отправить кадр, принадлежит не тому VLAN, который указан в кадре, то кадр будет уничтожен.

Пересылка кадров между сетевыми устройствами осуществляется с тэгом. Соединение, по которому пересылаются тэгированные кадры, называется \textit{транковой магистралью}. Интерфейсы, которые подключают конечные устройства, называются \textit{интерфейсами доступа}.

Посредством VLAN уменьшается широковещательный домен.

Для того, чтобы устройства из разных VLAN могли общаться между собой, необходим маршрутизатор. На нем помимо прочего можно настроить политику безопасности.

\image
{.5\textwidth}
{05/notes/inc/different-vlans}
{}

Один физический интерфейс логически разбивается на несколько: например, \texttt{fa0/0.1}, \texttt{fa0/0.2} и \texttt{fa0/0.3}.

\subsection{Протокол IPv4}

\begin{figure}[!htb]
    \centering
    \vphantom{\small1}
    \begin{bytefield}[bitwidth=0.03125\linewidth,bitformatting={\small}]{32}
        \bitheader{0,4,8,16,31}\\
        \bitbox{4}{Version} & \bitbox{4}{IHL} & \bitbox{8}{TOS} & \bitbox{16}{Total Length}\\
        \bitbox{16}{Identification} & \bitbox{3}{Flags} & \bitbox{13}{Fragment Offset}\\
        \bitbox{8}{TTL} & \bitbox{8}{Protocol} & \bitbox{16}{Header Checksum}\\
        \bitbox{32}{Source address}\\
        \bitbox{32}{Destination Address}\\
        \bitbox{32}{Options}
    \end{bytefield}
    \caption{Заголовок пакета IPv4}
    \label{img:ip}
\end{figure}

Поля заголовка пакета IPv4:

\begin{enumerate}
    \item \textbf{Номер версии}. Записана цифра 4.
    \item \textbf{IHL (Internet Header Length)}~--- длина заголовка. Считается в словах (1 слово~--- 4 байта). Длина заголовка~--- 5 (5 строк по 1 слову).
    \item \textbf{TOS (Type of Service)}~--- тип сервиса. Отвечает за качество обслуживания:
          \begin{enumerate}
              \item 0--2 биты~--- приоритет (precedence) данного IP-пакета. Позволяет нарушить логику FIFO в буфере интерфейса. Буфер разграничивается на три очереди по приоритетам: высокий, средний и низкий.
              \item 3 бит~--- требование ко времени задержки (delay) передачи IP-пакета. Работает, если на маршрутизаторе есть несколько маршрутов. Если бит выставлен, то пакет будет отправлен по маршруту с минимальной задержкой.
              \item 4 бит~--- требование к пропускной способности (throughput) маршрута, по которому должен отправляться IP-пакет. Если выставлен, то будет выбран маршрут, у которого выше пропускная способность.
              \item 5 бит~--- требование к надежности (reliability) передачи IP-пакета. Если выставлен, то будет выбран маршрут, у которого надежность выше (ниже процент потерянных пакетов).
              \item 6--7 биты~--- ECN (Explicit Congestion Notification)~--- явное сообщение о задержке. По отдельности эти биты <<не работают>>: они объединяются с двумя битами на транспортном уровне и так обеспечивают функционал. Позволяет маршрутизатору сообщить источнику данных, чтобы тот <<притормозил>> с передачей.
          \end{enumerate}
    \item \textbf{Длина пакета}. Измеряется в байтах.
    \item \textbf{Идентификатор}. Используется для идентификации фрагментов пакета, если он был фрагментирован. Все фрагменты одного пакета будут иметь одинаковый идентификатор.
    \item \textbf{Флаги}. Поле размером три бита, содержащее флаги контроля над фрагментацией. Биты, от старшего к младшему:
          \begin{enumerate}
              \item 1 бит~--- зарезервирован, должен быть равен 0.
              \item 2 бит~--- запрещена ли фрагментация.
              \item 3 бит~--- есть ли у пакета еще фрагменты. Должен быть установлен в 1 у всех фрагментов пакета, кроме последнего.
          \end{enumerate}
    \item \textbf{Смещение фрагмента}. Указывает смещение данных текущего фрагмента относительно исходных данных. Измеряется в количестве блоков по 8 байт.
    \item \textbf{TTL (Time to Live)}~--- время жизни. Определяет максимальное количество маршрутизаторов на пути следования пакета. Каждый маршрутизатор при обработке пакета должен уменьшить значение TTL на единицу. Пакеты, время жизни которых стало равно нулю, уничтожаются, а отправителю посылается сообщение ICMP Time Exceeded. Наличие этого параметра не позволяет пакету бесконечно ходить по сети (избегать \textit{петель маршрутизации}).
    \item \textbf{Протокол}. Обозначает протокол, которым необходимо передать содержимое поля данных.
    \item \textbf{Контрольная сумма}. Позволяет определить, были ли ошибки. Считается только по заголовку (данные не проверяются).
    \item \textbf{Адрес источника}~--- IP-адрес отправителя.
    \item \textbf{Адрес назначения}~--- IP-адрес получателя.
    \item \textbf{Опции}. Делятся на четыре класса:
          \begin{enumerate}
              \item 0 класс~--- датаграммы пользователя. Здесь может быть записан четкий маршрут, по которому должен быть отправлен пакет.
              \item 1 класс~--- зарезервирован, не используется.
              \item 2 класс~--- используется для отладки. Позволяет отслеживать маршрут, <<прося>> маршрутизаторы записывать, какое время пакет находился на конкретном маршрутизаторе.
              \item 3 класс~--- зарезервирован, не используется.
          \end{enumerate}
\end{enumerate}

\subsection{DHCP (Dynamic Host Configuration Protocol)}

IP-адрес на сетевом устройстве может появиться в двух случаях: если он был
задан вручную или получен автоматически. Во втором случае работает протокол
DHCP.

Протокол прикладного уровня (может работать, когда операционная система запущена)! Работает поверх UDP (Ethernet $\rightarrow$ IP $\rightarrow$ UDP $\rightarrow$ DHCP).

DHCP нужен для того, чтобы автоматически раздать устройствам IP-адреса.

\image
{.4\textwidth}
{05/notes/inc/dhcp}
{Процесс работы протокола DHCP}

Все запросы отправляются широковещательно. REQUEST отправляется широковещательно, потому что в сети может быть несколько DHCP-серверов. Таким образом сетевое устройство сообщает всем, какой из серверов был выбран. Серверы, которые не были выбраны, освобождают ресурсы. ACKNOWLEDGE отправляется широковещательно, потому что IP-адрес на сетевом устройстве назначается только после получения этого сообщения.

В результате работы протокола DHCP у сетевого устройства появляются: IP-адрес с маской, шлюз по умолчанию и (в идеале) адрес DNS-сервера.

\begin{dd}
    Шлюз по умолчанию (Default Gateway)~--- IP-адрес ближайшего интерфейса маршрутизатора. Если компьютеру необходимо отправить пакет за пределы своей сети, он отправляет его на этот адрес.
\end{dd}

DHCP-сервер может выдавать IP-адреса тремя способами.

\begin{enumerate}
    \item \textbf{Ручное назначение статических адресов}. Пары <<MAC-адрес~--- IP-адрес>> задаются вручную. Каждое устройство будет получать один и тот же IP-адрес в соответствии со своим MAC-адресом. Может применяться при настройке МФУ.
    \item \textbf{Динамическое назначение статических адресов}. На сервере задается пул адресов. Сервер автоматически создает пары <<MAC-адрес~--- IP-адрес>>. При последующих подключениях устройство будет получать один и тот же IP-адрес.
    \item \textbf{Автоматическое назначение динамических адресов}. Самый распространенный способ. Задается пул адресов и время аренды, в течение которого устройство может владеть IP-адресом. Сервер <<забывает>> про MAC-адрес устройства, как только оно отключается.
\end{enumerate}

\subsection{ARP (Address Resolution Protocol)}

Протокол сетевого уровня. Предназначен для разрешения IP-адреса в MAC-адрес. Нужен, чтобы на канальном уровне (в кадре Ethernet) выставлять MAC-адрес назначения.

\begin{figure}[!htb]
    \centering
    \vphantom{\small1}
    \begin{bytefield}[bitwidth=0.03125\linewidth,bitformatting={\small}]{32}
        \bitheader{0,8,16,31}\\
        \bitbox{16}{Hardware Type (HTYPE)} & \bitbox{16}{Protocol Type (PTYPE)}\\
        \bitbox{8}{\small Hardware Length (HLEN)} & \bitbox{8}{\small Protocol Length (PLEN)} & \bitbox{16}{Operation (OPER)}\\
        \bitbox{32}{Sender Hardware Address (SHA)}\\
        \bitbox{32}{Sender Protocol Address (SPA)}\\
        \bitbox{32}{Target Hardware Address (THA)}\\
        \bitbox{32}{Target Protocol Address (TPA)}
    \end{bytefield}
    \caption{Пакет ARP}
    \label{img:arp}
\end{figure}

Поля пакета ARP:

\begin{enumerate}
    \item \textbf{Тип адреса канального уровня}. Ethernet имеет номер \texttt{0x0001}.
    \item \textbf{Тип адреса сетевого уровня}. IPv4 имеет номер \texttt{0x0800}.
    \item \textbf{Длина адреса канального уровня}. Для Ethernet~--- \texttt{0x06} (длина MAC-адреса~--- 6 байт).
    \item \textbf{Длина адреса сетевого уровня}. Для IP~--- \texttt{0x04} (длина IP-адреса~--- 4 байта).
    \item \textbf{Код операции отправителя}. \texttt{0x0001} в случае запроса и \texttt{0x0002} в случае ответа.
    \item \textbf{MAC-адрес отправителя}.
    \item \textbf{IP-адрес отправителя}.
    \item \textbf{MAC-адрес получателя}. При запросе заполнено нулями.
    \item \textbf{IP-адрес получателя}. IP-адрес устройства, чей MAC-адрес ищется.
\end{enumerate}

ARP-запрос отправляется широковещательно, ответ будет unicast.

Существуют ARP-таблицы, записи в которых могут быть двух видов.

\begin{enumerate}
    \item \textbf{Статические записи}. Пары <<MAC-адрес~--- IP-адрес>> создаются вручную. Запросы не отправляются.
    \item \textbf{Динамические записи}. Пары создаются в процессе отправки и получения ARP-запросов и ответов.
\end{enumerate}

\subsection{RARP (Reverse ARP)}

При помощи протокола RARP ищется IP-адрес текущего устройства по MAC-адресу этого же устройства.

Может быть использован до загрузки операционной системы (и DHCP соответственно).

\subsection{VTP (VLAN Trunking Protocol)}

Протокол канального уровня. Предназначен для анонсирования VLAN. Придуман, чтобы избежать настройки нескольких VLAN на всех коммутаторах. Проприетарное решение Cisco (открытый аналог~--- GVRP).

Каждый коммутатор принимает одну из трех ролей (по умолчанию~--- сервер).

\begin{enumerate}
    \item \textbf{Сервер}. Разрешено создавать, переименовывать и удалять VLAN. Как только происходит какое-то действие с VLAN, сервер сообщает об этом клиентам своей сети.
    \item \textbf{Клиент}. Получает и применяет анонсы от серверов.
    \item \textbf{Прозрачный}. Получает, но не применяет анонсы от серверов, а пересылает их дальше.
\end{enumerate}

Анонсы <<ходят>> исключительно по магистральным соединениям.

Можно настроить домен и пароль на доступ к анонсам.

VTP работает в рамках одной сети, маршрутизаторы не пропускают через себя
анонсы.

