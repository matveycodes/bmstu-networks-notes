\section{Лекция №10 (27.11.2023)}

\subsection{IGRP (Interior Gateway Routing Protocol)}

Дистанционно-векторный проприетарный протокол Cisco. Аналогично RIP работает на трех таймерах, но с другими значениями:

\begin{itemize}
    \item таймер регулярной рассылки~--- 90 секунд;
    \item таймер таймаута~--- 270 секунд ($\times 3$);
    \item таймер сбора мусора~--- 630 секунд ($\times 7$).
\end{itemize}

Метрика:
%
\begin{gather*}
    M = \left(K_1\cdot B + \dfrac{K_2\cdot B}{256 - L} + K_3\cdot D\right)\cdot \dfrac{K_5}{K_4 + R}
\end{gather*}
%
Здесь:

\begin{enumerate}
    \item $B$ (bandwidth)~--- пропускная способность. Учитывается пропускная способность самого <<узкого>> канала.
    \item $L$ (load)~--- загрузка в процентном соотношении.
    \item $D$ (delay)~--- задержка в микросекундах.
    \item $R$ (reliability)~--- надежность (вероятность ошибки от 0 до 1).
\end{enumerate}

$K_i$~--- коэффициенты, которые можно самостоятельно настраивать. Значения по умолчанию:

\begin{itemize}
    \item $K_1 = K_3 = 1$;
    \item $K_2 = K_4 = K_5 = 0$.
\end{itemize}

При значениях по умолчанию значение метрики будет равно ${B + D}$.

\begin{figure}[!htb]
    \centering
    \vphantom{\small1}
    \begin{bytefield}[bitwidth=0.03125\linewidth,bitformatting={\small}]{32}
        \bitheader{0,4,8,16,23,31}\\
        \bitbox{4}{Version} & \bitbox{4}{OP Code} & \bitbox{8}{Edition} & \bitbox{8}{AS Number}\\
        \bitbox{16}{Number of Interior Routes} & \bitbox{16}{Number of System Routes}\\
        \bitbox{16}{Number of Exterior Routes} & \bitbox{16}{Checksum}\\
        \bitbox{24}{IP Address}\\
        \bitbox{24}{Delay}\\
        \bitbox{24}{Bandwidth}\\
        \bitbox{16}{MTU}\\
        \bitbox{8}{Reliability} & \bitbox{8}{Load} & \bitbox{8}{Hop Count}
    \end{bytefield}
    \caption{Пакет IGRP}
    \label{img:igrp}
\end{figure}

Поля пакета IGRP:

\begin{enumerate}
    \item \textbf{Версия}. Номер версии протокола.
    \item \textbf{Операционный код}. Определяет функционал сообщения:
          \begin{itemize}
              \item 1~--- пакет корректировки (отправляется при получении запроса и на регулярной основе);
              \item 2~--- запрос (чтобы не ждать истечения таймера регулярной рассылки).
          \end{itemize}
    \item \textbf{Номер версии}. Используется для контроля версий базы данных. Число, которое увеличивается на единицу при изменениях в сети.
    \item \textbf{Номер автономной системы}, которой принадлежит пакет. Логика аналогична OSPF: служебная информация передается в рамках одной автономной системы.
          \begin{dd}
              Автономная система~--- это совокупность зон (сетей), которые подчиняются одной логике администрирования.
          \end{dd}
    \item \textbf{Количество внутренних маршрутов}.
          \begin{dd}
              Внутренний маршрут~--- это маршрут, который был <<узнан>> протоколом, который рассылает обновления.
          \end{dd}
    \item \textbf{Количество системных маршрутов}.
    \item \textbf{Количество внешних маршрутов}.
          \begin{dd}
              Внешний маршрут~--- это маршрут, который был <<узнан>> другим протоколом.
          \end{dd}
    \item \textbf{Контрольная сумма}. Определяет наличие ошибок.
    \item \textbf{IP-адрес сети}. IGRP~--- классовый протокол, поэтому номер подсети не учитывается.
    \item[10--14.] \textbf{Параметры метрики}.

        \begin{dd}
            MTU (Maximum Transmission Unit)~--- максимальный объем данных, который можно передать по каналу за одну транзакцию.
        \end{dd}

        \addtocounter{enumi}{5}
    \item \textbf{Количество hop'ов}. Используется для устранения петель.
\end{enumerate}

\subsection{EIGRP (Enhanced IGRP)}

Используется на современном оборудовании.

Преимущества по сравнению с IGRP:

\begin{enumerate}
    \item \textbf{Быстрая сходимость}. Достигается за счет того, что маршрутизатор <<помнит>> про все возможные маршруты, а не только про оптимальные. Чтобы реализовать быструю сходимость, EIGRP работает с тремя таблицами:

          \begin{itemize}
              \item соседей;
              \item топологии;
              \item маршрутизации.
          \end{itemize}
    \item \textbf{Сниженное потребление полосы пропускания}. EIGRP отправляет регулярно только hello-сообщения, а не всю базу данных, как IGRP.
    \item \textbf{Отсутствие широковещательных сообщений}. Сообщения EIGRP отправляются на групповой адрес.
    \item \textbf{Балансировка между маршрутами неравной метрики}.
    \item \textbf{Автосуммирование} для уменьшения таблицы маршрутизации.

          \image
          {\linewidth}
          {10/notes/inc/eigrp}
          {Автосуммирование EIGRP}
\end{enumerate}