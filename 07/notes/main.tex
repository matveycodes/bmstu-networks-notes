\section{Лекция №7 (30.10.2023)}

\subsection{Протокол IPv6}

Отличия от IPv4:

\begin{itemize}
    \item увеличенное адресное пространство ($2^{128}$ против $2^{32}$);
    \item более простой заголовок;
    \item отсутствие широковещательной рассылки (и широковещательного адреса);
    \item Plug and Play (устройство само может сгенерировать себе IP-адрес без
          DHCP-серверов);
    \item обязательный IPsec.
\end{itemize}

\begin{figure}[!htb]
    \centering
    \vphantom{\small1}
    \begin{bytefield}[bitwidth=0.03125\linewidth,bitformatting={\small}]{32}
        \bitheader{0,4,8,16,31}\\
        \bitbox{4}{Version} & \bitbox{8}{Traffic Class} & \bitbox{20}{Flow Label}\\
        \bitbox{16}{Payload Length} & \bitbox{8}{Next Header} & \bitbox{8}{Hop Limit}\\
        \wordbox{4}[perword=\wordline]{Source Address}\\
        \wordbox{4}[perword=\wordline]{Destination Address}
    \end{bytefield}
    \caption{Заголовок пакета IPv6}
    \label{img:ipv6}
\end{figure}

\pagebreak
Поля заголовка пакета IPv6:

\begin{enumerate}
    \item \textbf{Версия}. То же, что и в IPv4. Записана цифра 6.
    \item \textbf{Приоритет пакета}. Аналог типа сервиса из IPv4.
    \item \textbf{Метка потока}. Присваивается узлом-отправителем путём генерации псевдослучайного 20-битного числа. Все пакеты одного потока должны иметь одинаковые заголовки, обрабатываемые маршрутизатором. При получении первого пакета с меткой потока маршрутизатор анализирует весь заголовок и запоминает результаты обработки в локальном кэше. Ключом для такой записи является комбинация адреса источника и метки потока. Последующие пакеты с той же комбинацией адреса источника и метки потока обрабатываются с учётом информации кэша без анализа всех полей заголовка.
    \item \textbf{Длина поля данных}. Измеряется в байтах.
    \item \textbf{Next Header}. Аналог поля протокола в IPv4.
    \item \textbf{Hop Limit}. Аналог поля TTL в IPv4.
    \item \textbf{Адрес источника}. IP-адрес источника.
    \item \textbf{Адрес получателя}. IP-адрес получателя.
\end{enumerate}

Ускорение обработки в протоколе IPv6 достигается за счет более простого заголовка и отсутствия необходимости пересчитывать контрольную сумму при изменении Hop Limit (как в случае с TTL в IPv4).

\subsection{Адресация IPv6}

IP-адрес в IPv6 занимает 16 байт и записывается в шестнадцатеричной системе
счисления:
%
\begin{center}
    \texttt{01ac:025d:0000:0000:0000:0000:0000:0250}\\
    \texttt{\phantom{0}1ac:\phantom{0}25d:\phantom{0000:0000:0000:0000:0000}:\phantom{0}250}\\
    \texttt{\phantom{0}1ac:\phantom{0}25d:\phantom{000}0:\phantom{000}0:\phantom{000}0:\phantom{000}0:\phantom{000}0:\phantom{0}250}
\end{center}
%
Допускается опускать ведущие нули, заменяя их двумя двоеточиями (не более одного раза!) или заменять 4 идущих подряд нуля одним нулем.

Разновидности IP-адресов в IPv6:

\begin{enumerate}
    \item \textbf{Unicast}. Рассылка <<один к одному>>. Unicast-адреса могут быть следующих видов:
          \begin{enumerate}
              \item Глобальный (белый) адрес. Выглядит как \texttt{2000::/3}.
              \item Локальный (серый) адрес. Делятся на:
                    \begin{enumerate}
                        \item Неопределенный. Выглядит как \texttt{::/128}. Назначается устройству до получения IP-адреса.
                        \item Локальный адрес. Выглядит как \texttt{::1/128}.
                        \item Локальный адрес канала. Выглядит как \texttt{fe80::/10}. Если в пакете записан такой адрес, то такой пакет не будет выходить за пределы маршрутизатора и будет передаваться внутри подсети (не всей локальной сети). Таким образом рассылается служебная информация.
                        \item Локальный адрес площадки. Выглядит как \texttt{fd::/8}. Пакеты с таким адресом <<ходят>> по всей локальной сети. Таким образом рассылается   пользовательская информация.
                    \end{enumerate}
          \end{enumerate}
    \item \textbf{Multicast}. Рассылка <<один ко многим>>. Аналог адресов класса D в IPv4.
    \item \textbf{Anycast}. Рассылка <<один к ближайшему>>. Пример: печать на принтере. NB! Ближайший не географически, а с точки зрения сети.
\end{enumerate}

\begin{figure}[!htb]
    \centering
    \vphantom{\small1}
    \begin{bytefield}[bitwidth=0.0078125\linewidth,bitformatting={\small}]{128}
        \bitheader{23,32,48,64,127}\\
        \bitbox{23}{Реестр} & \bitbox{9}{ISP} & \bitbox{16}{\small Префикс площадки} & \bitbox{16}{\small Префикс подсети} & \bitbox{64}{ID интерфейса}
    \end{bytefield}
    \caption{Глобальный unicast-адрес}
    \label{img:global-unicast}
\end{figure}

Структура глобального unicast-адреса IPv6:

\begin{enumerate}
    \item \textbf{Реестр}. Выдается IANA региональным регистраторам.
    \item \textbf{ISP}. Выдается национальным регистратором провайдерам.
    \item \textbf{Префикс площадки}. Выдается провайдером более малым провайдерам.
    \item \textbf{Префикс подсети}. Выдается малым провайдером клиентам.
    \item \textbf{ID интерфейса}. Уникальная комбинация, определяющая номер хоста.
\end{enumerate}

Если сравнивать с IPv4, то первые 64 бита~--- это адрес сети, а оставшиеся~--- адрес хоста.

\begin{figure}[!htb]
    \centering
    \vphantom{\small1}
    \begin{bytefield}[bitwidth=0.0078125\linewidth,bitformatting={\small}]{128}
        \bitheader{8,48,64,127}\\
        \bitbox{8}[bgcolor=lightgray]{FD} & \bitbox{40}{Global ID} & \bitbox{16}{\small Префикс подсети} & \bitbox{64}{ID интерфейса}
    \end{bytefield}
    \caption{Локальный unicast-адрес площадки}
    \label{img:local-site-unicast}
\end{figure}

Структура локального unicast-адреса площадки IPv6:

\begin{enumerate}
    \item \textbf{Global ID}. Псевдослучайное число, которое закрепляется за локальной сетью ее администратором. Все IP-адреса устройств, принадлежащих этой локальной сети, должны обладать одинаковым Global ID.
    \item \textbf{Префикс подсети}. В локальной сети может быть несколько подсетей~--- здесь указывается конкретная.
    \item \textbf{ID интерфейса}. Адрес хоста.
\end{enumerate}

\begin{figure}[!htb]
    \centering
    \vphantom{\small1}
    \begin{bytefield}[bitwidth=0.0078125\linewidth,bitformatting={\small}]{128}
        \bitheader{10,64,127}\\
        \bitbox{10}[bgcolor=lightgray]{FE80} & \bitbox{54}[bgcolor=lightgray]{0} & \bitbox{64}{ID интерфейса}
    \end{bytefield}
    \caption{Локальный unicast-адрес канала}
    \label{img:local-channel-unicast}
\end{figure}

Так как пакет с локальным адресом канала не выходит за пределы подсети, ему не нужно знать ни Global ID, ни префикс подсети.

ID интерфейса может быть назначен вручную, автоматически при помощи
DHCP-сервера или технологии EUI-64.

\subsection{EUI-64 (Extended Unique Identifier)}

64 бита для ID интерфейса формируются на основании MAC-адреса устройства. Для получения из 48-битного MAC-адреса 64-битного идентификатора первый делится пополам, а полученные части соединяются последовательностью \texttt{FFFE}.

\image
{.75\textwidth}
{07/notes/inc/eui-64}
{Процесс получения ID интерфейса при помощи EUI-64}

\subsection{Plug and Play}

Plug and Play работает следующим образом.

\begin{enumerate}
    \item Составляется адрес канала на основе MAC-адреса.
    \item Отправляется запрос идентификации в свою подсеть на групповой адрес всех маршрутизаторов.
    \item Маршрутизатор, получив пакет, представляется. Таким образом формируется адрес площадки: ID интерфейса есть, Global ID и префикс подсети получены из адреса маршрутизатора.
\end{enumerate}

\subsection{Маршрутизация}

Задача маршрутизации~--- найти оптимальный маршрут.

\begin{dd}
    Таблица маршрутизации~--- это таблица, в которой содержится информация об оптимальных маршрутах всех известных сетей.
\end{dd}

Существует два вида записей в таблице маршрутизации.

\begin{enumerate}
    \item \textbf{Статическая запись}. Создается вручную и не реагирует на изменения в реальном времени.
    \item \textbf{Динамическая запись}. Обновляется в реальном времени. Появляется из двух источников:
          \begin{enumerate}
              \item Напрямую подключенная сеть.
              \item Протокол маршрутизации.
          \end{enumerate}
\end{enumerate}

При работе протоколов маршрутизации путем обмена сообщениями между маршрутизаторами те строят оптимальные маршруты в своих таблицах маршрутизации. Если работает только один протокол маршрутизации, то он будет <<смотреть>> на \textit{метрику}.

\begin{dd}
    Метрика~--- это расстояние от данного маршрутизатора до сети назначения.
\end{dd}

Расстояние может измеряться различными способами в зависимости от протокола:

\begin{itemize}
    \item в количестве маршрутизаторов между сетями;
    \item на основании пропускной способности;
    \item на основании задержки;
    \item на основании надежности канала (вероятности ошибки);
    \item на основании загрузки канала.
\end{itemize}

\image
{.75\textwidth}
{07/notes/inc/topology}
{}

Первые три записи в таблице маршрутизации примера появятся без протокола маршрутизации (DC~--- Directly Connected).

В результате работы протокола маршрутизации появляется запись о седьмой сети, состоящая из следующих полей.

\begin{enumerate}
    \item \texttt{120} в \texttt{[120/2]}~--- административное расстояние, которое определяет степень доверия источнику информации. Изменяется от 0 до 255 (меньше~--- лучше). За каждым протоколом маршрутизации закреплено свое административное расстояние (Cisco):
          \begin{itemize}
              \item 0~--- DC-записи;
              \item 1~--- статические записи;
              \item \dots
              \item 120~--- RIP;
              \item \dots
          \end{itemize}
    \item \texttt{2} в \texttt{[120/2]}~--- значение метрики. В данном случае~--- количество маршрутизаторов между левым роутером и сетью правого компьютера.
    \item \texttt{192.168.3.2}~--- IP-адрес следующего маршрутизатора (Next Hop).
\end{enumerate}

\subsection{Параметры протоколов маршрутизации}

При выборе протокола маршрутизации следует учитывать следующие параметры.

\begin{enumerate}
    \item \textbf{Оптимальность алгоритма}. Зависит от того, как медленно вычисляется оптимальный маршрут. Оптимальный маршрут зависит от метрики. Метрика зависит от показателей (пропускная способность, надежность, и т. д.). В некоторых протоколах можно выставить вес каждого параметра.
    \item \textbf{Низкие непроизводительные затраты}. Так как через маршрутизаторы проходит много трафика, протокол маршрутизации должен обеспечить свой функционал с минимальными непроизводительными затратами.
    \item \textbf{Стабильность работы}. Так как маршрутизаторы находятся в ключевых точках сети, протокол маршрутизации должен уметь обрабатывать ситуации, когда они выходят из строя. Сюда же относится балансировка нагрузки.
    \item \textbf{Быстрая сходимость}. При долгой сходимости пакеты начинают зацикливаться и теряться.
          \begin{dd}
              Сходимость~--- это соглашение между всеми маршрутизаторами сети об оптимальных маршрутах.
          \end{dd}
\end{enumerate}
