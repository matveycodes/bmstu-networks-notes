\section{Лекция №2 (11.09.2023)}

Весь функционал модели OSI/ISO реализован при помощи служебной информации~--- PDU.

\begin{dd}
    Protocol Data Unit (PDU)~--- это одиночный блок информации, передаваемый в компьютерной сети. Состоит из протоколо-зависимой информации и данных пользователя~--- заголовка (обеспечивает функционал), данных и хвостовика (контрольная последовательность).
\end{dd}

PDU создается на каждом уровне модели OSI/ISO.

\image
{\textwidth}
{02/notes/inc/incapsulation}
{Поток данных в модели OSI/ISO}

Процесс упаковки~--- инкапсуляция. Обратный процесс~--- деинкапсуляция.

Названия PDU:

\begin{itemize}
    \item транспортный уровень~--- сегмент (segment);
    \item сетевой уровень~--- пакет (packet);
    \item канальный уровень~--- кадр (frame).
\end{itemize}

Интернет не работает по OSI/ISO, он работает по DOD (Department of Defense). В этой модели не семь уровней, а четыре:

\begin{etaremune}
    \item Прикладной (верхние три уровня OSI/ISO).
    \item Транспортный.
    \item Сетевой.
    \item Канальный (физический и канальный уровни OSI/ISO).
\end{etaremune}

Как результат~--- меньше заголовков. Стек протоколов TCP/IP.

Модель $\neq$ стек протоколов: модель~--- сущность, описывающая функционал, а стек протоколов~--- просто набор протоколов.

При проектировании сети следует руководствоваться трехуровневой иерархической структурой, включающей в себя уровень доступа, уровень распределения, уровень ядра:

\image
{\textwidth}
{02/notes/inc/three-level-hierarchy}
{Трехуровневая иерархическая модель сети}

Уровень доступа предоставляет соединение конечных устройств. На этом уровне работают канальные (L2-) устройства.

Уровень распределения обеспечивает качество обслуживания, безопасность, соединение разных уровней доступа (слиение множества доступов в одно соединение). Здесь могут работать L2- и L3-устройства.

Уровень ядра~--- самый нагруженный уровень: здесь весь трафик сети. Здесь работают L3-устройства.

\subsection{Физический уровень}

Кабель называется \textit{витой парой}. Он называется так, потому что под изоляцией~--- четыре пары перевитых проводков. Ток, идущий по медной жиле, создает помехи и наводки на соседние провода. Чтобы это исправить, провода перевивают.

Для прокладки локальных сетей используют кабели категории 5Е. Категории отличаются количеством проводов внутри и скоростью передачи данных.

Шестая категория кабелей имеет экран, который защищает соседние кабели от наводок.

Два варианта обжимки: европейский и американский. Для обжимки используется специальный инструмент~--- кримпер.

Коннектор RJ45.

Есть три типа обжима витой пары.

\begin{enumerate}
    \item \textbf{Прямой}. С двух сторон~--- одинаковый стандарт. Используется для соединения устройств на разных уровнях моделей OSI/ISO (компьютер~--- коммутатор, коммутатор~--- маршрутизатор).
    \item \textbf{Перекрестный}. С двух сторон~--- разные стандарты. Используется для соединения двух одинаковых устройств (компьютер~--- компьютер, маршрутизатор~--- маршрутизатор, компьютер~---маршрутизатор (!)).
    \item \textbf{Зеркальный}. С одной стороны~--- любой из стандартов, с другой~--- его зеркальное отражение. Используется для соединения по консоли (для настройки оборудования).
\end{enumerate}

\subsection{Канальный уровень}

\begin{dd}
    MAC-адрес~--- физический адрес устройства. Состоит из 48 битов, записывается в шестнадцатеричной системе счисления. Есть у любого устройства в сети. Должен быть уникален.
\end{dd}

MAC-адрес состоит из двух частей: ID организации (0--23 биты) и ID устройства
(24--47 биты):
%
\begin{gather*}
    \underbrace{\strut\mathtt{00:1B:44}}_{\text{ID организации}}\mathtt{:}\underbrace{\strut\mathtt{11:3A:B7}}_{\text{ID устройства}}
\end{gather*}
%
\subsubsection{Token Ring (маркерное кольцо)}

Топология: кольцевая.\\
Метод: детерминированный.

В классической реализации работает на двух скоростях~--- 4 и 16 Мбит/с.

В Token Ring на 4 Мбит/с PDU создается отправителем после полного прохождения круга предыдущей PDU, на 16 Мбит/с~--- сразу после отправки предыдущей PDU (раннее освобождение маркера).

\begin{dd}
    Активный монитор~--- станция, имеющая самый большой MAC-адрес.
\end{dd}

Активный монитор создает первую PDU, контролирует отсутствие PDU без получателя. Каждые 3 секунды отправляет кадр своей активности (<<я живой!>>). Через 7 секунд неактивности выбирается новый активный монитор.

В кольце существуют три вида PDU:

\begin{itemize}
    \item маркер;
    \item кадр данных;
    \item прерывающая последовательность.
\end{itemize}

Маркер~--- PDU, которую надо <<захватить>>. 3 байта, байт-ориентированный, состоит из трёх частей.

\image
{\textwidth}
{02/notes/inc/token-ring-pdu}
{}

\begin{enumerate}
    \item Начальный ограничитель (JK~--- специально внесённые ошибки~--- манчестерский код).
    \item Управление доступом:
          \begin{enumerate}
              \item Биты P~--- биты приоритета. Максимальное значение~--- ${2^3-1=7}$. Маркер может захватить только та станция, у которой приоритет данных для передачи больше или равен значению этих бит.
              \item Бит T~--- бит маркера. Если выставлен, то имеем дело с маркером, иначе~--- с кадром данных.
              \item Бит M~--- бит монитора. Выставляется активным монитором либо при создании PDU, либо при прохождении PDU через него. <<Сбивается>> любой другой станцией, видоизменяющей PDU.
              \item Биты R~--- биты резервного приоритета. Используются для того, чтобы <<записаться>> в очередь на передачу. Записаться может только та станция, у которой приоритет данных для передачи больше значения этих бит.
          \end{enumerate}
    \item Конечный ограничитель:
          \begin{enumerate}
              \item Бит I~--- информационный бит. Указывает, последний ли это кадр в последовательности.
              \item Бит E~--- бит ошибки (была найдена ошибка или нет).
          \end{enumerate}
\end{enumerate}

Кадр данных состоит из следующих частей.

\image
{\textwidth}
{02/notes/inc/token-ring-frame}
{}

\begin{enumerate}
    \item Начальный ограничитель.
    \item Управление доступом.
    \item Управление кадром. Определяет функционал PDU:
          \begin{enumerate}
              \item Передача пользовательской информации (любой информации с уровня выше).
              \item Передача служебной информации (та информация, которая позволяет корректно работать протоколу). Например, поиск нового активного монитора, обнаружение разрыва.
          \end{enumerate}
    \item DA (Destination Address)~--- MAC-адрес получателя\footnote{Адрес получателя
              идет перед адресом источника для повышения скорости обработки кадров на
              канальном уровне. На всех остальных уровнях порядок обратный. Это связано с
              тем, что L2-устройства обрабатывают весь трафик, а L3+-устройства~--- только
              тот, который им предназначается.}.
    \item SA (Source Address)~--- MAC-адрес источника.
    \item Data~--- данные.
    \item FCS (Frame Control Sequence)~--- контрольная последовательность кадра.
          Позволяет определить, была ли ошибка (аналог контрольной суммы). При наличии
          ошибки будет выставлен бит E в конечном ограничителе.
    \item Конечный ограничитель.
    \item Статус кадра~--- \texttt{ACxxACxx}\footnote{Если источник получает свой кадр данных со значениями A = 1, C = 1, E = 1, то он считает, что данные дошли корректно (ошибка возникла после копирования получателем при передаче дальше по кольцу).}:
          \begin{enumerate}
              \item Биты A~--- был ли распознан получатель. Как только данные доходят до получателя (в DA станция <<видит>> свой MAC-адрес), он выставляет биты A.
              \item Биты C~--- были ли данные скопированы.
          \end{enumerate}
\end{enumerate}

Прерывающая последовательность <<чистит>> кольцо (например, при смене активного монитора) и состоит из двух частей.

\begin{enumerate}
    \item Начальный ограничитель.
    \item Конечный ограничитель.
\end{enumerate}

Физически кольцо организовывается при помощи специальных концентраторов, которые бывают активными и пассивными.

Пассивный концентратор замыкает реле, когда станция выключается (происходит её обход при передаче кадров).

Активный концентратор, помимо замыкания реле, усиливает сигнал.

\image
{\textwidth}
{02/notes/inc/token-ring-physical}
{Физическая организация маркерного кольца}
