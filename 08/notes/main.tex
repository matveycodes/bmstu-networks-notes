\section{Лекция №8 (13.11.2023)}

\subsection{Классы протоколов маршрутизации}

Первая классификация:

\begin{enumerate}
    \item \textbf{Статические}. Работает за счет записей, внесенных вручную. Работает быстро. Не подстраивается в реальном времени.
    \item \textbf{Динамические}. Работают за счет обмена обновлениями между маршрутизаторами. Работает дольше. Подстраивается в реальном времени.
\end{enumerate}

Вторая классификация:

\begin{enumerate}
    \item \textbf{Одномаршрутные}. Оптимальный маршрут всегда один.
    \item \textbf{Многомаршрутные}. Оптимальных маршрутов может быть несколько (разные маршруты с одинаковой метрикой или метрикой в допустимом диапазоне считаются оптимальными). Маршрутизатор будет балансировать трафик между этими маршрутами двумя способами:
          \begin{itemize}
              \item по принципу Round Robin;
              \item на основании транзакций (пакеты от одного источника одному получателю будут проходить по одному маршруту).
          \end{itemize}
\end{enumerate}

Третья классификация:

\begin{enumerate}
    \item \textbf{Одноуровневые}. Все маршрутизаторы знают про все сети. Рассылка обновлений происходит по всем маршрутизаторам локальной сети.
    \item \textbf{Иерархические}. Существуют базовые и вспомогательные маршрутизаторы. Сеть делится на нулевую (backbone) зону и остальные зоны, которые подключается к ней. Служебная информация пересылается внутри зоны (за редким исключением). Маршрутизаторы знают только о сетях зоны, в которой находятся.
\end{enumerate}

Четвертая классификация:

\begin{enumerate}
    \item \textbf{Дистанционно-векторные}. Маршрутизаторы обмениваются таблицами маршрутизации только с <<соседями>>.
    \item \textbf{Состояния канала}. Обмен идет только о состояниях собственных каналов (таблицы маршрутизации не рассылаются) между всеми маршрутизаторами. Каждый маршрутизатор самостоятельно рассчитывает оптимальные маршруты.
\end{enumerate}

\subsection{RIP (Routing Information Protocol)}

Дистанционно-векторный протокол для маршрутизации внутри локальной сети. Административное расстояние~--- 120. Работает на основе трех таймеров.

\begin{enumerate}
    \item \textbf{Таймер регулярной рассылки}. По истечении таймера каждый маршрутизатор отправляет свою таблицу маршрутизации каждому своему <<соседу>>. Значение по умолчанию~--- 30 секунд.
    \item \textbf{Таймер таймаута}. Запускается для каждой записи в таблице маршрутизации в момент ее появления. Обновляется, когда приходит обновление с этой записью. По истечении таймера запись считается предположительно недостижимой, запускается таймер сбора мусора. Значение по умолчанию~--- 180 секунд.
    \item \textbf{Таймер сбора мусора}. По истечении таймера предположительно недостижимая запись удаляется из таблицы маршрутизации. Значение по умолчанию~--- 120 секунд.
\end{enumerate}

В качестве метрики RIP использует количество <<хопов>> (промежуточных устройств). Максимально допустимое количество <<прыжков>>~--- 15. Метрика со значением 16 считается бесконечной и назначается предположительно недостижимым сетям.

Существует три версии протокола.

\subsubsection{RIP v1}

\begin{figure}[!htb]
    \centering
    \vphantom{\small1}
    \begin{bytefield}[bitwidth=0.03125\linewidth,bitformatting={\small}]{32}
        \bitheader{8,16,31}\\
        \bitbox{8}{Команда} & \bitbox{8}{Версия} & \bitbox{16}[bgcolor=lightgray]{0}\\
        \bitbox{16}{ID адресного семейства} & \bitbox{16}[bgcolor=lightgray]{0}\\
        \bitbox{32}{IP-адрес сети}\\
        \bitbox{32}[bgcolor=lightgray]{0}\\
        \bitbox{32}[bgcolor=lightgray]{0}\\
        \bitbox{32}{Метрика}
    \end{bytefield}
    \caption{Пакет RIP v1}
    \label{img:ripv1}
\end{figure}

Структура пакета RIP первой версии (передает одну запись маршрутизации):

\begin{enumerate}
    \item \textbf{Команда}. Определяет функционал сообщения:
          \begin{enumerate}
              \item Запрос. Маршрутизатор при включении отправляет запрос на получение таблицы маршрутизации (чтобы не ждать 30 секунд регулярной рассылки). Пакет запроса состоит из первой строки.
              \item Ответ на запрос или сообщение регулярной рассылки.
          \end{enumerate}
    \item \textbf{Версия}. Номер версии протокола. Записана цифра 1.
    \item \textbf{ID адресного семейства}. Для расширяемости.
    \item \textbf{IP-адрес сети}.
    \item \textbf{Метрика}.
\end{enumerate}

\subsubsection{RIP v2}

Причина создания второй версии~--- невозможность первой работать с бесклассовой адресацией (нет маски).

\begin{figure}[!htb]
    \centering
    \vphantom{\small1}
    \begin{bytefield}[bitwidth=0.03125\linewidth,bitformatting={\small}]{32}
        \bitheader{8,16,31}\\
        \bitbox{8}{Команда} & \bitbox{8}{Версия} & \bitbox{16}[bgcolor=lightgray]{0}\\
        \bitbox{16}{ID адресного семейства} & \bitbox{16}{Тэг маршрута}\\
        \bitbox{32}{IP-адрес сети}\\
        \bitbox{32}{Маска}\\
        \bitbox{32}{IP-адрес Next Hop}\\
        \bitbox{32}{Метрика}
    \end{bytefield}
    \caption{Пакет RIP v2}
    \label{img:ripv2}
\end{figure}

Структура пакета RIP второй версии:

\begin{enumerate}
    \item \textbf{Тэг маршрута}. Предназначен для разделения <<внутренних>> и <<внешних>> маршрутов, взятых, например, из другого IGP или EGP.
    \item \textbf{Маска}. Нужна для бесклассовой адресации.
    \item \textbf{IP-адрес Next Hop}. Содержит IP-адрес маршрутизатора к месту назначения. Значение \texttt{0.0.0.0}~--- хопом к месту назначения является отправитель пакета. Необходимо, если протокол RIP не может быть запущен на всех маршрутизаторах.
\end{enumerate}

Во второй версии сообщение рассылается на групповой адрес \texttt{224.0.0.9}, а не на широковещательный. Если в сети есть устройства, работающие с разными версиями RIP, то вторая версия сможет работать посредством обновлений первой, а первая посредством второй~--- нет.

\subsection{Петли маршрутизации}

\begin{dd}
    Петля маршрутизации~--- это ситуация, при которой пакет зацикливается между какими-то маршрутизаторами.
\end{dd}

\image
{\textwidth}
{08/notes/inc/loop}
{Пример петли маршрутизации}

Пакет будет находиться в петле, пока не TTL не станет равным 0.

Существует пять методов борьбы с петлями маршрутизации.

\begin{enumerate}
    \item \textbf{Расщепление горизонта}. Информация не отправляется назад источнику информации.
    \item \textbf{Отравление маршрута}. Отключенный маршрут будет включаться в обновление с бесконечной метрикой.
    \item \textbf{Обратное отравление маршрута}. Маршрутизатору-источнику информации маршрут отправляется с бесконечной метрикой.
    \item \textbf{Таймер удержания}. Запускается в момент отключения маршрута. В течение работы этого таймера не применяются обновления о данном маршруте, если метрика равна или хуже прежней.
    \item \textbf{Триггерные сообщения}. Рассылаются маршрутизаторами при изменении сети сразу после того, как они зафиксировали это изменение.
\end{enumerate}