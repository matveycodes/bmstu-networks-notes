\section{Лекция №9 (20.11.2023)}

\subsection{OSPF (Open Shortest Path First)}

Протокол состояния канала~--- все маршрутизаторы сети обмениваются обновлениями. В обновление включается состояние только собственных каналов маршрутизатора. На основе этой информации каждый маршрутизатор строит граф, по которому самостоятельно ищет оптимальный маршрут.

Административное расстояние~--- 110.

В качестве метрики используется пропускная способность:
%
\begin{gather*}
    M = \dfrac{\text{100 Мбит/с}}{\text{пропуск. способность}}
\end{gather*}
%
Поиск оптимального маршрута осуществляется с помощью алгоритма Дейкстры.

OSPF~--- иерархический протокол: зоны...

\begin{figure}[!htb]
    \centering
    \vphantom{\small1}
    \begin{bytefield}[bitwidth=0.03125\linewidth,bitformatting={\small}]{32}
        \bitheader{0,8,16,31}\\
        \bitbox{8}{Версия} & \bitbox{8}{Тип} & \bitbox{16}{Длина сообщения}\\
        \bitbox{32}{ID маршрутизатора}\\
        \bitbox{32}{ID области}\\
        \bitbox{16}{Контрольная сумма} & \bitbox{16}{Тип идентификации}\\
        \wordbox{2}[perword=\wordline]{Идентификация}
    \end{bytefield}
    \caption{Заголовок пакета OSPF}
    \label{img:ospf}
\end{figure}

Поля заголовка пакета OSPF:

\begin{enumerate}
    \item \textbf{Версия}. Номер версии протокола. Существует три версии: первая <<забыта>> в конце 90-х, вторая используется для IPv4, третья~--- для IPv6.
    \item \textbf{Тип}. Определяет функционал сообщения. Существует пять разновидностей:
          \begin{enumerate}
              \item \textbf{Hello}. Единственный вид сообщения, который рассылается на регулярной основе. Каждый маршрутизатор каждые 5 секунд отправляет это сообщение, показывая, что он активен.
              \item \textbf{Database Description}. База данных, хранящая описание всех каналов сети: перечень маршрутизаторов, сетей и их пропускных способностей. На основании этой базы данных маршрутизатор строит граф и ищет оптимальный маршрут.
              \item \textbf{Link State Request}. Запрос, который новый маршрутизатор в сети отправляет DR, чтобы получить базу данных.
              \item \textbf{Link State Update}. Сообщение генерируется в двух случаях:
                    \begin{itemize}
                        \item произошло изменение в сети (триггерное сообщение);
                        \item подтверждение записи в базе данных по запросу (незадолго до истечения таймера).
                    \end{itemize}
              \item \textbf{Link State Acknowledgment}. Отправляется маршрутизатором, который получил Database Description.
          \end{enumerate}
    \item \textbf{Длина сообщения}. Так как отправляемые сообщения отличаются по длине (например, hello-сообщение и сообщение с базой данных), существует это поле.
    \item \textbf{ID маршрутизатора}. Используется для идентификации маршрутизаторов. Выглядит так же, как и IP-адрес. В локальной сети существует две роли маршрутизаторов:
          \begin{itemize}
              \item DR (Designated Router)~--- главный маршрутизатор;
              \item BDR (Backup Designated Router)~--- запасной главный маршрутизатор.
          \end{itemize}
          Маршрутизатор с наименьшим ID становится DR. Следующий по старшинству ID маршрутизатор становится BDR.

          ID назначается двумя способами:
          \begin{enumerate}
              \item Вручную.
              \item Автоматически на основании IP-адресов логических интерфейсов: в качестве ID будет выбран минимальный IP-адрес среди всех логических интерфейсов маршрутизатора. Если логические интерфейсы не настроены, по такому же принципу будут рассмотрены физические интерфейсы.
          \end{enumerate}
    \item \textbf{ID области}. Определяет, какой зоне принадлежит сообщение.
    \item \textbf{Контрольная сумма}. Проверяет наличие ошибок.
    \item \textbf{Идентификация}. Существует три типа:
          \begin{itemize}
              \item \textbf{Без идентификации}. Сообщение может быть обработано любым маршрутизатором.
              \item \textbf{Открытым текстом}. Пароль будет передан <<как есть>>. Сообщение будет обработано маршрутизатором, пароль которого совпадает с паролем в сообщении.
              \item \textbf{MD5}. Пароль будет захэширован алгоритмом MD5. Сообщение будет обработано маршрутизатором, хэш пароля которого совпадает с хэшом пароля в сообщении.
          \end{itemize}
\end{enumerate}

<про wildcard-маску>

