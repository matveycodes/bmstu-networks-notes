\section{Лекция №3 (18.09.2023)}

\subsubsection{FDDI (Fiber Distributed Data Interface)}

Разработан на основе Token Ring.

Использует детерминированный метод доступа к физической среде, работает по модели кольца.

В отличие от Token Ring работает на оптике. Алгоритм только раннего освобождения маркера. Скорость начинается от 100 Мбит/с. Использует два кольца одновременно~--- более отказоустойчив.

<...>

Локальные сети строят для:

\begin{itemize}
    \item совместного использования данных;
    \item совместного использования ресурсов;
    \item доступа к глобальным сетям (ISP).
\end{itemize}

Группы компонентов локальной сети:

\begin{itemize}
    \item сетевые устройства (маршрутизаторы, коммутаторы, концентраторы);
    \item соединительные устройства (провода, Wi-Fi);
    \item конечные устройства (ПК, смартфоны, планшеты);
    \item протоколы.
\end{itemize}

\subsubsection{Соединительные устройства}

Изначально локальные сети строили на коаксиальных кабелях.

Витая пара категории 5E при передаче 100 Мбит/с даёт длину сегмента 90 м. Если больше, то при большом объёме трафика будут потери пакетов. Решить эту проблему можно при помощи сетевых устройств~--- усилителей, концентраторов.

Между конечными устройствами не может быть больше четырех хабов~--- \textit{правило четырех хабов}. Максимальное расстояние~--- 450 м. Правило объясняется тем, что даже на первый хаб при длине сегмента 90 м сигнал приходит затухшим, и при его восстановлении возникают ошибки. Чем больше хабов, тем больше ошибок.

Оптические кабели бывают одномодовые и многомодовые.

\image
{.75\textwidth}
{03/notes/inc/fiber-mode}
{}

\begin{dd}
    Домен коллизии~--- это устройства, которые конкурируют за одну среду передачи данных.
\end{dd}

Коммутатор~--- более <<умное>> по сравнению с концентратором устройство канального уровня (L2-устройство). Его задача~--- <<разбить>> домен коллизии.

\subsection{Ethernet II}

Протокол канального уровня.

\image
{\textwidth}
{03/notes/inc/ethernet-frame}
{Кадр Ethernet (размер в байтах)}

Заголовок кадра состоит из трех полей.

\begin{enumerate}
    \item DA (Destination Address)~--- MAC-адрес получателя.
    \item SA (Source Address)~--- MAC-адрес источника.
    \item Type~--- код протокола, которым необходимо передать содержимое поля Data.
\end{enumerate}

Коммутатор прослушивает все, что через него проходит. Благодаря \textit{матрице коммутаций} одновременно могут передавать $N$ устройств (вместо одной среды передачи~--- $N$). В результате домен коллизии <<разбивается>>.

\begin{dd}
    Микросегментация~--- это свойство коммутатора создавать маршруты внутри себя.
\end{dd}

Интерфейсы коммутатора нумеруются по стандарту. Например, \texttt{fa0/1}:

\begin{itemize}
    \item \texttt{fa}~--- fast ethernet (100 Мбит/с);
    \item \texttt{0}~--- номер слота;
    \item \texttt{1}~--- номер интерфейса в этом слоте.
\end{itemize}

\begin{dd}
    Таблица коммутации~--- таблица сопоставлений номера интерфейса и MAC-адреса устройства, находящегося на этом интерфейсе.
\end{dd}

Таблицы в сетевых устройствах бывают статическими (заполняются вручную, записи не обладают временем жизни) и динамическими (заполняются и обновляются автоматически).

\subsection{Виды коммутации}

Существует три вида коммутации (один из них зашит в коммутаторе~--- поменять невозможно).

\begin{enumerate}
    \item \textbf{С буферизацией}. Самая надежная, но самая медленная коммутация. Весь кадр помещается в буфер, анализируется на наличие коллизий и ошибок и в случае успеха отправляется получателю.
    \item \textbf{Без буферизации}. Самая быстрая, но наименее надежная коммутация. В буфер попадает только первое поле (DA), после чего сразу создается маршрут.
    \item \textbf{Бесфрагментный}. В буфер попадают первые 64 байта, анализируются на коллизии (но не на ошибки).
\end{enumerate}

\subsection{Функции коммутатора}

Коммутатор обладает двумя функциями.

\begin{enumerate}
    \item \textbf{Пересылка}. В случае, если из таблицы коммутации известен получатель, то кадр отправляется только ему.
    \item \textbf{Лавинная рассылка (flooding)}. Если местонахождение получателя неизвестно, то коммутатор отправляет кадр на все интерфейсы, кроме того, с которого получил.
    \item \textbf{Фильтрация}. Если кадр предназначается тому же сегменту, из которого пришел, то он уничтожается.
\end{enumerate}

\subsection{Разновидности коммутаторов}

Коммутаторы делятся на три разновидности.

\begin{enumerate}
    \item \textbf{SOHO (Small Office Home Office)}. Объем таблицы коммутации~--- от \num{1000} до \num{8000} записей.
    \item \textbf{SMB (Small Medium Business)}. Объем таблицы~--- \num{64000} записей.
    \item \textbf{Для провайдеров}. Объем таблицы~--- \num{128000} записей.
\end{enumerate}

В случае, если память заканчивается, то затирается самая старая запись.

На одном интерфейсе может быть столько MAC-адресов, сколько есть памяти на таблицу коммутации.