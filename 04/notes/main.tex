\section{Лекция №4 (25.09.2023)}

% Есть специальный вид коммутаторов~--- стекируемые коммутаторы. На их задних панелях есть специальный стековый интерфейс для соединения коммутаторов в стек. В стек можно соединить двумя топологиями~--- простым деревом или кольцом. При этом стековые интерфейсы позволяют слить таблицу коммутации нескольких коммутаторов в единую таблицу. Эти интерфейсы должны иметь более высокую скорость, чем интерфейсы для подключения конечного оборудования.

\subsection{Сетевой уровень}

В качестве адреса~--- не физический адрес, а логический~--- IP-адрес. В этом адресе <<зашита>> информация о том, кто где находится.

\subsubsection{Адресация IPv4}

В общем случае IP-адрес состоит из двух частей: адреса сети и адреса хоста.

IPv4 занимает 32 бита, записывается в десятичной системе счисления в виде
четырех октетов:
%
\begin{center}
    \texttt{192.168.1.1}
\end{center}
%
Существует два подхода к определению того, где заканчивается адрес сети и начинается адрес хоста: классовый и бесклассовый.

\subsubsection{Классовый подход}

Существует пять классов. В зависимости от класса определяется граница.

\begin{enumerate}
    \item \textbf{Класс A}. Граница~--- после первого октета. Первый бит~--- \texttt{0}. Сети~--- с 0 по 127.
    \item \textbf{Класс B}. Граница~--- после второго октета. Первые биты~--- \texttt{10}. Сети~--- с 128 по 191.
    \item \textbf{Класс C}. Граница~--- после третьего октета. Первые биты~--- \texttt{110}. Сети~--- с 192 по 223.
    \item \textbf{Класс D}. Не обладает границами. Адреса зарезервированы под групповую рассылку
          (адреса из этого класса закреплены за конкретными протоколами) и используются
          для передачи служебной информации. Например, отправить информацию всем маршрутизаторам,
          работающим по протоколу OSPF. Первые биты~--- \texttt{1110}. Сети~--- с 224 по 239.
    \item \textbf{Класс E}. Не обладает границами. Адреса зарезервированы для дальнейшей разработки. Используется только последний адрес~--- \texttt{255.255.255.255}. Первые биты~--- \texttt{1111}. Сети~--- с 240 по 255.
\end{enumerate}

Есть определенный вид IP-адресов, которые используются под определенные цели (под другие использовать не получится):

\begin{itemize}
    \item адрес сети~--- \texttt{С.С.С.0} (биты хостовой части <<забиваются>> нулями)~--- используется для работы маршрутизаторов;
    \item локальный широковещательный адрес~--- \texttt{255.255.255.255}~--- используется для отправки пакетов всем устройствам данной сети;
    \item направленный широковещательный адрес~--- \texttt{С.С.С.255} (все хостовые биты <<забиваются>> единицами)~--- используется для отправки пакетов всем устройствам данной подсети;
    \item адрес автоконфигурации~--- \texttt{169.254.X.X}~--- для автоматического получения IP-адреса в случае недоступности DHCP-сервера;
    \item неопределенный адрес~--- \texttt{0.0.0.0}~--- присваивается устройству при включении сетевой карты;
    \item локальный адрес~--- \texttt{127.X.X.X}.
\end{itemize}

Устройств больше, чем доступных уникальных адресов IPv4, поэтому их разделили на частные и публичные. Публичные (<<белые>>) адреса~--- это те адреса, которые обслуживаются в интернете. Частные (<<серые>>) адреса не обслуживаются за пределами локальной сети. В каждом из классов выделили пул серых адресов (остальные~--- белые).

\begin{enumerate}
    \item Класс А~--- с \texttt{10.0.0.0} по \texttt{10.255.255.255}.
    \item Класс B~--- с \texttt{172.16.0.0} по \texttt{172.31.255.255}.
    \item Класс C~--- с \texttt{192.168.0.0} по \texttt{192.168.255.255}.
\end{enumerate}

\subsubsection{NAT}

NAT (Network Address Translation) меняет серый IP-адрес на белый и наоборот при
прохождении пакета через маршрутизатор.

Существует три технологии NAT.

\begin{enumerate}
    \item \textbf{Статический NAT}. Прописываем <<руками>> какой IP-адрес~--- серый, а какой~--- белый. <<Статика>> используется, чтобы, например, сделать сервер доступным из локальной сети. Сколько записей, столько и устройств могут выходить в интернет.
    \item \textbf{Динамический NAT}. Задается пул белых адресов. Маршрутизатор, принимая пакеты из локальной сети, назначает им адрес из этого пула. Сколько адресов в пуле, столько и устройств могут выходить в интернет.
    \item \textbf{Port Address Translation (PAT)}. Самый распространенный вид NAT. Каждому пакету назначается белый IP-адрес, соответствующий интерфейсу выхода в глобальную сеть.
\end{enumerate}

\subsubsection{Маска}

Как и IP-адрес занимает 32 бита, делится на четыре октета.

Там, где в маске~--- единицы, там~--- адрес сети. Там, где в маске~--- нули, там~--- адрес хоста.
%
\begin{center}
    \texttt{192.168.\phantom{11}1.1}\\
    \texttt{255.255.255.0}
\end{center}
%
Чередование нулей и единиц недопустимо.

Есть второй вариант записи маски:
%
\begin{center}
    \texttt{192.168.1.1/24}
\end{center}
%
\subsubsection{Бесклассовая адресация}

Есть два варианта (алгоритма) бесклассовой адресации~--- VLSM (Variable Length Subnet Mask) и CIDR (Classless Inter-Domain Routing).

В VLSM IP-адрес состоит из трех частей: сети, подсети и хоста.

<Рисует задачи на VLSM (нужно учитывать 2 зарезервированных адреса!)>