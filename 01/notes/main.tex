\section{Лекция №1 (04.09.2023)}

\subsection{Методы передачи данных}

\begin{enumerate}
    \item \textbf{Симплексный}. Передача строго в одном направлении. Примеры: передача на принтер, телевидение, радио.
    \item \textbf{Полудуплексный}. Передача в обе стороны попеременно. Пример: рация.
    \item \textbf{Дуплексный}. Передача в обе стороны одновременно. Пример: телефонная связь.
\end{enumerate}

Метод передачи информации сильно зависит от применяемого оборудования. При этом можно перейти с более высокого метода на более низкий, но не наоборот.

\subsection{Протоколы}

Передача данных происходит с помощью \textit{протоколов}.

\begin{dd}
    Протокол~--- это набор соглашений логического уровня.
\end{dd}

Протоколы бывают без установления соединения и с установлением соединения.

\subsubsection{Протоколы без установления соединения}

Передающее оборудование не <<спрашивает>> у получателя, готов ли тот принять сообщение.

Минус~--- нет подтверждения, что данные были получены (получатель может быть выключен).

Протоколы без установления соединения применяются для передачи \textit{синхронного} (чувствительного к задержкам) трафика: например, видео в реальном времени.

\subsubsection{Протоколы с установлением соединения}

Отправитель убеждается в том, что принимающее оборудование готово к получению сообщения. При этом получатель всегда сообщает источнику, что данные дошли корректно~--- надежная передача. Если данные не дошли, то источник об этом узнает и повторит передачу.

Протоколы с установлением соединения применяются, когда необходимо что-либо надежно доставить: например, электронные письма, файлы, сообщения в мессенджерах.

Протоколы также бывают бит- и байт-ориентированными.

\subsubsection{Бит-ориентированные}

Местоположение каждого бита строго определяет его функционал.

\subsubsection{Байт-ориентированные}

Присутствуют управляющие комбинации, которые регламентируют передачу.

Проблема: данные где-то могут выглядеть точно так же, как управляющая комбинация. Решения:

\begin{itemize}
    \item запретить данным принимать вид управляющих комбинаций~--- непрозрачная передача;
    \item данные могут выглядеть как управляющая комбинация~--- прозрачная передача.
\end{itemize}

При прозрачной передаче используют следующие исправления.

\begin{enumerate}
    \item \textbf{Байт-стаффинг} (логический). Если в данных встречается управляющая комбинация, то перед этой управляющей комбинацией вставляется байт определенного вида, который удаляется на получателе.
    \item \textbf{Бит-стаффинг} (логический). Если в данных встречается пять единиц подряд, то после пятой 1 уставливается 0, который удаляется на получателе.
    \item \textbf{Неповторимые управляющие комбинации} (физический). Управляющие комбинации намеренно содержат ошибки~--- недопустимые сигналы. Например, если в середине такта должен быть переход из 0 в 1 или наоборот, то неизменный уровень сигнала является ошибкой.
\end{enumerate}

\subsection{Сети}

Сети делятся на локальные (LAN~--- Local Area Network) и глобальные (WAN~--- Wide Area Network).

Глобальная сеть~--- совокупность соединенных между собой локальных сетей.

\subsection{Топологии}

\begin{dd}
    Топология~--- это граф, вершины которого~--- сетевые устройства, а ребра~--- каналы связи.
\end{dd}

\begin{dd}
    Топология~--- это способ организации устройств в сеть (то, как устройства соединены между собой).
\end{dd}

Существует всего пять топологий, все остальные~--- их производные.

\subsubsection{Шина}

Исторически самая первая топология.

\image{.5\textwidth}{01/notes/inc/bus}{}

Изначально локальные сети строили на коаксиальном кабеле. У коаксиального кабеля одна жила. Сигнал идет во все стороны, на концах его нужно глушить при помощи \textit{терминаторов}.

Основной минус~--- при разрыве кабеля <<падает>> вся сеть.

\subsubsection{Кольцо}

Все устройства соединены в кольцо.

\image{.375\textwidth}{01/notes/inc/ring}{}

Минус~--- при разрыве кабеля <<падает>> вся сеть.

Плюс~--- при использовании протокола без установления соединения отправитель все равно получает подтверждение доставки.

\subsubsection{Звезда}

Наиболее распространенная на данный момент топология.

\image{.375\textwidth}{01/notes/inc/star}{}

Плюс~--- при разрыве одного кабеля сеть продолжит работать.

Минус~--- если из строя выходит центральный узел, то сеть <<падает>>.

\subsubsection{Полносвязная}

Самая надежная топология.

\image{.875\textwidth}{01/notes/inc/full}{}

Минусы:

\begin{itemize}
    \item цена (каждое соединение~--- сетевой адаптер);
    \item проблема размещения всех кабелей;
    \item широковещательные шторма;
    \item нестабильность таблиц, помогающих сетевым устройствам понимать, кто где находится;
    \item одна и та же информация будет приходить с разных сторон в виде копий.
\end{itemize}

Обычно логически блокируют избыточные (резервные) каналы.

\subsubsection{Частично связная}

Одно устройство (например, сервер) соединяется со всеми, а остальные~--- произвольно.

Топологии бывают физическими и логическими.

\subsubsection{Физическая}

Физическая топология~--- это то, как соединены устройства.

\subsubsection{Логическая}

Логическая топология~--- это то, как передается трафик.

При этом физическая и логическая топологии могут не совпадать. Например, Wi-Fi: физическая топология~--- звезда, логическая~--- шина.

\subsection{Дисциплины передачи информации}

Дисциплины передачи информации определяют, кому и когда можно вещать.

Есть две дисциплины: иерархическая и одноранговая.

\subsubsection{Иерархическая}

В иерархической дисциплине есть один первичный и множество вторичных узлов. Все вторичные подчиняются первичному.

Если у первичного узла есть данные для передачи вторичному, то он сообщает вторичному о необходимости передачи информации, тот отвечает, затем идет передача.

Если у вторичного узла есть данные для передачи первичному, то вторичный <<молчит>> до тех пор, пока первичный не <<спросит>> у него, есть ли что-то для передачи. Первичный периодически опрашивает каждую станцию. Вторичный начинает передачу только тогда, когда ему <<разрешили>>. Таким образом, одновременного вещания нескольких устройств в сети нет.

Передача между вторичными узлами осуществляется через первичный.

\subsubsection{Одноранговая}

Все узлы равны между собой. Может работать на <<шине>> или <<кольце>>. Все имеют право вещать, когда <<захотят>>. Возникают коллизии.

Чтобы их не было, на <<шине>> работает технология CSMA/CD (Carrier Sense Multiple Access with Collision Detection~--- множественный доступ с контролем несущей и обнаружением коллизий).

Множественный доступ: благодаря технологии обеспечивается доступ к физической среде (каналу передачи данных) многим устройствам.

Контроль несущей: станции прослушивают канал и начинают передачу, если он свободен.

\image{.5\textwidth}{01/notes/inc/wave}{}

Обнаружение коллизий: из-за того, что существуют задержки распространения сигналов, два или более устройств могут начать передачу данных одновременно, что приведет к коллизиям. В этом случае сигнал оказывается непохожим на те, которые описаны в протоколе, и устройства понимают, что произошла коллизия. Передающая станция (станции) при обнаружении коллизии отправляет jam-сигнал, чтобы усилить коллизию, и запускает таймер на псевдослучайный промежуток времени. По истечении времени попытка передачи повторяется.

В <<кольце>> работает детерминированный метод. Его суть в том, что каждому устройству выделяется квант времени, в течение которого разрешена передача данных. По кольцу <<бегает>> специальный блок, который надо <<захватить>>, поместить в него свои данные и передать дальше.

\subsection{OSI/ISO}

Из-за того, что каждый производитель сетевого оборудования делал устройства, несовместимые с устройствами других производителей, было невозможно создавать мультивендорные сети. Для решения этой проблемы международной организацией по стандартам (ISO) была разработана \textit{эталонная модель взаимодействия открытых систем} (OSI/ISO).

\begin{etaremune}
    \item \textbf{Прикладной уровень}.

    Связывает всю модель OSI с GUI. Предоставляет услуги самому приложению.

    \item \textbf{Уровень представления}.

    Отвечает за корректное отображение данных на получателе: например, работа с кодировкой.

    \item \textbf{Сеансовый уровень}.

    Предназначен для установления, поддержания и корректного завершения сеанса. Отвечает за аутентификацию.

    \pagebreak
    \item \textbf{Транспортный уровень}.

    Отвечает за:

    \begin{itemize}
        \item сегментацию данных на отправителе и организацию данных в поток на получателе;
        \item надёжную передачу данных (установление соединения, вычисление и проверка контрольных сумм).
    \end{itemize}

    Самые распространённые протоколы транспортного уровня~--- \textsc{TCP} (с соединением) и \textsc{UDP} (без соединения).

    \item \textbf{Сетевой уровень}.

    Предназначен для поиска и выбора оптимального маршрута между географически удаленными сетями.

    Устройства: маршрутизаторы.

    \item \textbf{Канальный уровень}.

    Первый уровень, который определяет формат данных для передачи. Отвечает за метод контроля доступа к физической среде (CSMA/CD и т. д.). На этом уровне обеспечивается надежная передача (контрольная последовательность кадра).

    Устройства: мосты\footnote{Больше не используются.}, коммутаторы.

    \item \textbf{Физический уровень}.

    Отвечает за механические, электрические, процедурные и функциональные характеристики установления, поддержания и завершения соединения между физическими устройствами.

    Устройства: медиаконвертеры, сетевые адаптеры, усилители, концентраторы.
\end{etaremune}