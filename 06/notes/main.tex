\section{Лекция №6 (23.10.2023)}

\subsection{STP (Spanning Tree Protocol)}

Протокол связующего дерева.

Протокол позволяет уйти от петель на канальном уровне, так как нет TTL. Петли возникают, когда в топологию вводится избыточность. При отправке широковещательного запроса (например, ARP-) возникнет широковещательный шторм.

Протокол STP борется со следующими проблемами:

\begin{itemize}
    \item широковещательные шторма;
    \item нестабильность таблиц коммутации (постоянная перезапись);
    \item множественные копии кадров.
\end{itemize}

Построение дерева протоколом STP начинается с выбора корневого моста. Корневной мост выбирается на основании Bridge ID, который состоит из двух частей: приоритета моста и MAC-адреса моста. Коммутатор с минимальным значением Bridge ID является корнем. Приоритет по умолчанию~--- \num{32768}.

\image
{.75\textwidth}
{06/notes/inc/stp}
{Bridge ID в протоколе STP}

После того, как корень найден, выбираются \textit{корневые порты}. Корневой порт выбирается на коммутаторе и предоставляет минимальное расстояние от данного коммутатора до корневого моста. Расстояние высчитывается на основании стоимости, а стоимость~--- на основании пропускной способности. Чем больше пропускная способность, тем лучше маршрут.

Когда корневой порт найден, на сегменте ищется \textit{назначенный порт}. Назначенный порт предоставляет минимальное расстояние от данного сегмента до корневого моста. Назначенный порт будет у того коммутатора, у которого меньше Bridge ID.

После выбора всех назначенных портов оставшиеся интерфейсы переходят в состояние \textit{заблокированных}.

Перед тем, как порт начнет передавать пользовательские данные, он проходит через несколько состояний.

\begin{enumerate}
    \item \textbf{Blocking}. Через 30 секунд уходит в Listening.
    \item \textbf{Listening}. Порт передает служебные кадры протокола. Через 15 секунд уходит в состояние Learning.
    \item \textbf{Learning}. Порт передает служебные кадры и строит таблицу коммутации, анализируя (но не передавая) пользовательские данные. Через 15 секунд переходит в Forwarding (если назначена <<передающая>> роль).
    \item \textbf{Forwarding}. Передача пользовательских и служебных данных.
\end{enumerate}

Из-за задержки в 50 секунд был разработан протокол RSTP.

\subsection{RSTP (Rapid STP)}

Логика построения дерева та же.

Состояния Blocking и Listening объединены в Discarding. Таким образом, состояний всего три, а протокол сходится за 30 секунд.

В RSTP появилась роль порта PortFast (назначается вручную). Интерфейс с этой ролью не будет участвовать в просчете дерева (сразу перейдет в Forwarding). Этот интерфейс должен смотреть на конечное оборудование.

Роль Blocking в RSTP поделена на альтернативный и резервный. Разница в том, что альтернативный интерфейс в будущем станет корневым портом, а резервный~--- назначенным.

\subsection{PVSTP (Per VLAN STP) / MSTP (Multiple STP)}

Аналогичен RSTP, но корневые мосты могут различаться для разных VLAN. Появляется возможность балансировки трафика.

\subsection{ICMP (Internet Control Message Protocol)}

Протокол сетевого уровня. В отличие от всех остальных протоколов сетевого уровня инкапсулируется в IP.

Цели~--- сообщения об ошибках и диагностика сети.

\begin{figure}[!htb]
    \centering
    \vphantom{\small1}
    \begin{bytefield}[bitwidth=0.03125\linewidth,bitformatting={\small}]{32}
        \bitheader{0,8,16,31}\\
        \bitbox{8}{Type} & \bitbox{8}{Code} & \bitbox{16}{Checksum}\\
        \wordbox{1}{Content}
    \end{bytefield}
    \caption{Пакет ICMP}
    \label{img:icmp}
\end{figure}

Поля заголовка пакета ICMP:

\begin{enumerate}
    \item \textbf{Тип}. Определяет функционал сообщения.
    \item \textbf{Код}. Конкретизирует функционал сообщения (дополнительная контекстная информация).
    \item \textbf{Контрольная сумма}. Ищет ошибки. Вычисляется на основании заголовка и данных.
\end{enumerate}

Протокол ICMP используется в команде \texttt{tracert} (\texttt{traceroute}) и позволяет отследить маршрут пакета. В случае с \texttt{tracert} используется TTL, который проходит значения от 0 до 255.

ICMP-ответы не будут генерироваться в следующих ситуациях.

\begin{enumerate}
    \item Уничтожение ICMP-пакета.
    \item Уничтожение широковещательной рассылки.
    \item Потеря не первого фрагмента фрагментированного пакета.
\end{enumerate}

<Рисует топологии и объясняет ARP>

